\label{Grundlagen der IT-Sicherheit}
%\begin{section}{Grundlagen der IT-Sicherheit}
 In diesem Kapitel werden allgemeine Begriffe wie zum Beispiel die Grundbegriffe der
 Sicherheit, diverse Sicherheitsziele aus dem Grundschutzkatalog sowie Schutzbedarf 
 Bedrohungsanalysen behandelt. 
 \\
 Weiters wird eine klare Grenze zwischen Sicherheit und IT-Sicherheit gezogen. 
 Abschließend wird auf das Teilgebiet der VoIP-Sicherheit eingegangen.
 \\
%\end{section}

 \label{Sicherheitsziele}
  \begin{section}{Sicherheitsziele}
  Das deutsche Bundesamt für Sicherheit in der Informationstechnik (BSI) beschreibt 
  folgende Sicherheitsziele (security objectives) in einem Sicherheitskatalog: 
  \cite{BSIBedarf}
  \begin{itemize}
   \item Verfügbarkeit (availability)
   \item Integrität (integrity)
   \item Vertraulichkeit (privacy, confidentialty)
  \end{itemize}
  \pagebreak
  
   \begin{subsection}{Verfügbarkeit}
    Verfügbarkeit (availability) meint die Eigenschaft eines Systems, innerhalb eines 
    bestimmten Zeitraums mit einer bestimmten Wahrscheinlichkeit die vom eingesetzten 
    System erwarteten Anforderungen zu erfüllen. Die Verfügbarkeit zählt zu den 
    Qualitätsmerkmalen einer IT-Landschaft.
    \\
   \end{subsection}

   \label{Integrität}
   \begin{subsection}{Integrität}
    Unter der Integrität (integrity) von Informationen versteht man die Vollständigkeit und 
    Korrektheit (Unversehrtheit) der übertragenen Daten auf Sender- und auch Empfängerseite. 
    Dabei wird davon ausgegangen, dass keine Manipulation der Daten auf dem Transportweg 
    durchgeführt wurden und somit unveränderte Daten verschickt wurden. Um diese 
    Datenintegrität an beiden Kommunikationsenden zu kontrollieren, kann man verschiedene 
    Kontrollmethoden verwenden: \ac{Z.B.} Hash-Verfahren, message authentification codes 
    oder digitale Signaturen. Um diesen Mechanismus durchführen zu können, sind die Inhalte 
    der Daten mit den Kontrollmethoden direkt verknüpft. 
    Durch unzureichende Integritätsprüfung können fehlerhafte Datensätze entstehen: Z.B. 
    Fehlerhafte Produktionen, Falsche Warenlieferungen oder Verbuchungsfehler. 
    In den letzten Jahren wird dem Verlust an Authentizität als Teil der Datenintegrität 
    verstärktes Augenmerk gewidment. 
    Im schlimmsten Fall wirken sich Fehler auf Zahlungen und Identitäten aus, welche an 
    nicht berechtigte Personen ausgestellt werden. So kann es beispielsweise zu 
    Identitätsdiebstahl kommen.
    \cite{BSIInfosi} \cite{BSIa} \cite{BSIBegriffe1}
    \\
   \end{subsection}

   \label{Vertraulichkeit}
   \begin{subsection}{Vertraulichkeit}
    Um eine sichere und effektive Datenübertragung und Verarbeitung zu gewährleisten, muss 
    eine hohe Vertraulichkeit (confidentiality) erreicht werden und erhalten bleiben. 
    Allgemein lässt sich die Vertraulichkeit als Gewährleistung beschreiben, bei welcher die 
    zu verarbeitenden Daten nur an jene Personen zugestellt und verfügbar gemacht werden, die 
    auch die nötigen Berechtigungen besitzen. Auf der anderen Seite werden Personen mit 
    keinerlei Berechtigung von den Daten separiert.
    \\
    \end{subsection}

    \label{Zuverlässigkeit}
    \begin{subsection}{Zuverlässigkeit}
     Microsoft definiert sechs Merkmale, welche auf einem System angewandt werden sollen. 
     Diese Qualitätsmerkmale bestehen im Wesentlichen aus den oben genannten Komponenten 
     Verfügbarkeit und Integrität.
     \begin{enumerate} 
      \item ausfallsicher \\
       Das System kann dem Benutzer auch dann Dienste anbieten, wenn eine interne oder 
       externe Unterbrechung stattfindet. (Verfügbarkeit)
      \item wiederherstellbar \\
       Das System kann nach einer benutzerbedingten oder systembedingten Unterbrechung 
       mittels Instrumentation oder Diagnose problemlos wieder ohne Datenverlust in seinen 
       ursprünglichen Zustand zurückgeführt werden. (Verfügbarkeit)
      \item kontrolliert \\
       Das System erfüllt im Bedarfsfall immer korrekt und rasch die Anforderungen an den 
       gewünschten Dienst. (Verfügbarkeit, Integrität)
      \item unterbrechungsfrei \\
       Erforderliche Änderungen und Aktualisierungen unterbrechen den Systembetrieb nicht. 
       (Verfügbarkeit)
      \item produktionsbereit \\
       Das System weist bereits bei der Auslieferung nur ein Minimum an Fehlern auf, 
       wodurch nur eine begrenzte Zahl an vorhersehbaren Aktualisierungen nötig ist. 
       (Verfügbarkeit)
      \item berechenbar \\
       Das System funktioniert wie erwartet bzw. wie vereinbart. Die Funktionalität von 
       früher bleibt weiter erhalten. (Verfügbarkeit, Integrität) \\
      \cite{Micro05}
     \end{enumerate}
    \end{subsection}
  \end{section}

 \label{Schutzbedarf}
 \begin{section}{Schutzbedarf}
  Der Schutz von Daten und Informationsflüssen gehört für die Unternehmen und öffentlichen
  Einrichtungen von heute zu den heikelsten und wichtigsten Aufgaben. Ständig aktuelle 
  Daten sind ein unerlässliches Gut für Unternehmen und sollten ständig auf dem neuesten 
  Stand sein. Um diese Daten-Updates durchführen zu können, werden möglichst viele 
  Informationen auf Datenträgern persistent gespeichert. Dabei kann sichergestellt werden, 
  dass diese Daten keinem Risiko ausgesetzt sind sondern mittels spezieller Methodik 
  gesichert und verwahrt werden. Andererseits gibt es aber auch Daten, welche nicht oder 
  nur schwach gesichert werden. Dies kann im schlimmsten Fall zum Verlust von Daten und 
  erheblichen Geldbeträgen führen. Grundlegende Sicherheitsmaßnahmen sind mithilfe von 
  geringen Mitteln und Maßnahmen zu erreichen. \\
  \\
  Der Schutzbedarf "Hoch" bedeutet, dass für die Abdeckung dieser Schutzkategorien 
  spezielle Maßnahmen ergriffen werden müssen. 
  Das \ac{BSI} stellte eine Zusammenstellung aus IT-Grundschutz-Vorgehensweisen und IT-Grundschutzkatalogen
  an, welche die Voraussetzungen für den Einsatz in den unterschiedlichsten Umgebungen 
  darstellt. \\
  \\
  Der Trend zur Speicherung aller Arten von Daten gehört genauso zum alltäglichen Leben 
  wie der Computer selbst. Von Vorratsdatenspeicherung bis hin zu Onlinedatenspeicherung 
  vernetzt über den Globus betrifft das Thema fast jeden Menschen. Selbst der Staat setzt 
  bei Speicherung von Patientendaten und Steuerdaten auf Informationsvernetzung und - 
  speicherung. \\
  \\
  Um den teilweise öffentlichen Forderung an Konsistenz und Sicherheit nachkommen zu 
  können, muss der Sicherheitsgrad hoch und der Schutzbedarf gedeckt sein. 
  Schutzbedarf selbst ist nicht quantifizierbar - aber das \ac{BSI} teilt diesen in drei 
  Kategorien: \\
  \begin{enumerate} 
   \item Normal \\
    Die Schadensauswirkungen sind begrenzt und überschaubar.
   \item Hoch \\
    Die Schadensauswirkungen können beträchtlich sein.
   \item Sehr Hoch \\
    Die Schadensauswirkungen können ein existentiell bedrohliches bzw. katastrophales 
    Ausmaß erreichen. \cite{BSIBedarf}
  \end{enumerate}
  \pagebreak

  Die Sicherheit von Daten und Systemen beschränkt sich heute nur auf technische 
  Gegebenheiten sondern umfasst auch die Sicherheit von organisatorischen und personellen 
  Umgebungsvariablen. Bei diesen ist es enorm wichtig, dass man behutsam und mit großer 
  Flexibilität ans Werk geht. Des Weiteren ist es wichtig, dass mit der richtige Umgang 
  mit sensiblen Daten und mit Betriebsvariablen sichergestellt wird. \\
  \\
  Für die heutigen Entwicklungsschritte in der Vernetzungstechnologie sind folgende 
  Faktoren prägend: \\
  
  \begin{enumerate} 
   \item Vernetzung \\
    Das Phänomen der Vernetzung wurden in den letzten Jahren durch Internet, VoIP und 
    viele andere technische Errungenschaften verstärkt. Die rasante Entwicklung in den 
    letzten zwei Jahrzehnten hat gezeigt, dass heute nicht mehr isoliert und lokal 
    gebunden gearbeitet wird, sondern mit verschiedenen Rechner und Menschen rund um den 
    Globus kommuniziert wird. Die globale Vernetzung bietet viele verschiedene 
    Möglichkeiten für verteiltes Arbeiten. So können etwa verteilte Ressourcen, gemeinsam 
    genutzte Datenbestände und Cloudcomputing zum eigenen Vorteil genutzt werden. \\
 
   \item IT-Verarbeitung und Durchdringung \\
 
 
   \item Schwindende Netzgrenzen \\
 
 
    \cite{BSIa}
   \end{enumerate}
 \end{section}

 \begin{section}{IT-Schutzkataloge}
  \cite{BSIITkat}
 \end{section}

 \label{Firewalls}
 \begin{section}{Firewalls}

  \label{Sicherheitsdienste einer Firewall}
  \begin{subsection}{Sicherheitsdienste einer Firewall}
  \end{subsection}
 
  \label{Firewall-Konzepte}
  \begin{subsection}{Firewall-Konzepte}
  \end{subsection}
 \end{section}
\pagebreak
