\label{Grundlagen von VoIP}
% \begin{section}{Grundlagen von VoIP}
	Allgemein gesprochen beschreibt Voice over IP die Telefonie basierend auf dem \ac{TCP/IP} Protokoll im Internet.
	
	Mit VoIP werden mehrere Begriffe verbunden; z.B.:
	\begin{itemize}
		\item \ac{IP}-Telefonie
		\item Skype
		\item \ac{LAN}-Telefonie
		\item \ac{SIP}-Telefonie
		\item Internettelefonie
	\end{itemize}

 \label{Geschichte der Telefonie}
 \begin{section}{Geschichte der Telefonie}
 	Kommunikation ist ein Grundbedürfnis aller Menschen. Allein wie die Information übermittelt wird, 
 	hat sich im Laufe der Geschichte immer wieder geändert. 
 	Seit Jahrzehnten ist das Telefon fester Bestandteil unseres Lebens. 
 	Der Griff zum Hörer, der Gespräche mit Menschen rund um die Welt ermöglicht, 
 	ist heute eine Selbstverständlichkeit. Vor 12 Jahren, 
 	als die Geschichte der Telefonie in Österreich begann, war das Telefon jedoch eine sensationelle Innovation. \\
 	\\
 	Im Juni des Jahres 1881 erteilte das k.k. Handelsministerium der "Wiener Privat-Telegraphen-Gesellschaft" 
 	eine "Concession" zum Betrieb von Telefonanlagen. Zwar erscheint die Netzabdeckung aus 
 	heutiger Sicht wenig beeindruckend, die Telefonanlagen durften lediglich in einem Umkreis 
 	von 15 Kilometern rund um den Wiener "Stephansthurm" betrieben werden. 
 	Schon drei Monate später wurde der Betrieb der Telefonanlagen auf ganz Wien ausgeweitet, 
 	im Dezember 1881 konnte in der Wiener Friedrichstraße die erste Telefonzentrale Österreichs 
 	eröffnet werden. Mit 154 Teilnehmern, darunter Zeitungen, Großunternehmer und Banken, 
 	wurde der Netzbetrieb gestartet. \\
 	\\
 	Im Jahr 1867 entwickelte Alexander Graham Bell eine Versuchsanordnung, 
 	bei der Schallwellen der menschlichen Sprache eine Membran vibrieren ließen. 
 	Dadurch entstanden in einer Drahtspule Stromschwankungen, 
 	die nach der Übertragung auf ein gleichartiges Gerät wieder Töne hervorbrachten. 
 	Bell erhielt für seine Erfindung die Patentnummer 174.465 und war damit nur um zwei Stunden 
 	schneller als der Amerikaner Elisha Gray. Damals konnte noch niemand ahnen, 
 	dass die Erfindung zu einem der einträglichsten Patente aller Zeiten werden sollte. 
 	Im Jahr 1876 wurde über den "Bellschen Sprachtelegraphen" schließlich zwischen 
 	Boston und Cambridge das erste Ferngespräch der Welt geführt.
 	\pagebreak
 	
 	Im Jahre 1910 kamen schließlich die ersten Telefone, die mit einer Wählscheibe und 
 	einem Hörer ausgestattet waren, auf den Markt. Zugleich bereitete die 
 	Post- und Telegraphenverwaltung mit der Umstellung der Verbindungen innerhalb eines Ortes 
 	vom handvermittelten Dienst auf Selbstwählverkehr eine kleine Revolution vor. 
 	Nun konnten die Teilnehmer erstmals ihre Gesprächspartner innerhalb eines Ortes 
 	direkt und ohne Vermittlungshilfe erreichen. Schon in den Anfangsjahren der Telefonie 
 	gab es Verbindungen in die österreichischen Kronländer. Allerdings wurden diese Leitungen 
 	bis etwa 1920 ausnahmslos über Freileitungstrassen geführt. \\

 	\cite{stadtwien2012Telefon}
 	\pagebreak
 \end{section}
 
 \label{Einführung in VoIP}
 \begin{section}{Einführung in VoIP}
 	Mit der Thematik Voice over IP hat die Telefonie einen riesen Schritt vorwärts getan.
 	Durch die starke Verbreitung von Smartphones haben sich noch weitere Geschäftsfelder für 
 	die Internettelefonie hinzugekommen. Internationale Firmen haben diese Chancen längst erkannt 
 	und stellen sich dementsprechend auf. \\
 	Basierend auf den Protokollen \ac{IP}, \ac{TCP} und \ac{UDP} lässt sich VoIP leicht in bestehende 
 	Infrastruktur und Dienste integrieren.
 	Im nächsten Kapitel werden die technischen Details noch näher erläutert. \\
 	Firmen nutzen die gleiche Infrastruktur zur Datenübertragung und zur Telefonie.
 	Das brachte viele Vorteile mit sich und somit auch den Durchbruch für VoIP.
 	Herkömmliche Telefonie war davor immer kanalorientiert gewesen. 
 	Im Bereich von VoIP spricht man von einer Paketorientierung im Bezug 
 	auf die Verbindung. \\
	Eine paketorientierte Übertragung von Sprache nutzt Leitungskapazität 
	gezielter aus als die bisherige Methode einer Reservierung der gesamten Leitung.
	Durch Effizienz in der Mediumsnutzung und Flexibilität in der Anwendung ergab sich 
	im Laufe der Zeit eine Überlegenheit von VoIP. \\
	\todo[inline]{Quelle hinzufügen}
	Als Konsequenz daraus ergibt sich eine hohe Erreichbarkeit.
	Gebremst werden kann das nur durch die verfügbare Bandbreite, welche sich 
	aber im Laufe der letzten Jahre verbessert hat. Das aufkommende Datenvolumen 
	spielt in seltenen Fällen eine Rolle - wird jedoch von den Providern kritischer 
	gesehen. \\
	Wesentlich für das Zustandekommen von Internettelefonie und Gesprächen ist, 
	dass die analoge Telefonie digital verarbeitet wird. 
	Die Sprache wird in IP-Pakete umgewandelt und in manchen Fällen auch verschlüsselt, 
	um die Sicherheit zu erhöhen.
	
 	\todo[inline]{Quelle hinzufügen}
 	
 	\textbf{Vorteile von VoIP}
 	\begin{itemize}
 		\item Interne Gespräche ohne Gebühren
 		\item Daten und Sprachübertragung in einem Medium
 		\item Ortsungebundenheit im Gegensatz zum Festnetz
 		\item Kostenersparnis bei Auslandsgesprächen
 		\item Erweiterbarkeit
 		\item Unabhängige Komplettlösung
 	\end{itemize}
 	
 	\textbf{Nachteile von VoIP}
 	\begin{itemize}
 		\item Sicherheitsrisiko durch Abhängigkeit von Internet
 		\item Qualitätsdefizit möglicherweise durch Hardware und Bandbreite
 		\item Zuverlässigkeit fraglich durch Hardwaredefekt oder Leitungsausfall
 		\item Notrufe nur eingeschränkt möglich
 		\item Stromkosten wesentlich höher
 		\item Externe Abhängigkeit von regionaler Infrastruktur (z.B. Stromnetz)
 	\end{itemize}
 	\todo[inline]{Quelle hinzufügen}
 	 	
 \end{section}
 
 \label{Technische Grundlagen einer VoIP-Infrastruktur}
 \begin{section}{Technische Grundlagen einer VoIP-Infrastruktur}
	Bei der Internettelefonie gibt es mehrere Schwerpunkte, die in diesem Kapitel 
	diskutiert werden.
	Die Kernfragen sind der Transport der Sprache, die Sicherung von Datenströmen 
	und Sprachinformationen und der Medienübergang in angrenzende Systeme.
 	\label{Protokolle und Standards}
 	\begin{subsection}{Protokolle und Standards}
 		\todo[inline]{RTP und RTCP}
 		\todo[inline]{H.323}
 		\todo[inline]{SIP}
 		\todo[inline]{SIP - STP}
 	\end{subsection}
 
 	\label{Anforderungen an die technische Infrastruktur}
 	\begin{subsection}{Anforderungen an die technische Infrastruktur}
 	\end{subsection}
 
	 \label{Komponenten}
 	\begin{subsection}{Komponenten}
 	\end{subsection}
 
 \end{section}
 
 \label{Fallbeispiel: Exemplarischer Aufbau einer VoIP-Infrastruktur}
 \begin{section}{Fallbeispiel: Exemplarischer Aufbau einer VoIP-Infrastruktur}
 \end{section}
 
% \end{section}
